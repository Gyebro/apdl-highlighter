\documentclass[magyar]{apdl}
\usepackage[utf8]{inputenc}
\usepackage{fancyvrb}

\begin{document}

\codefile{proba.txt}

\title{Téglalap hálózása}
\date{2018/08/15}
\course{Bevezetés a síkidomok hálózásába}
\tasknum{1/b}

\maketitle

\task{Itt kezdődik a feladatkitűzés. Vegyünk egy $a=3\unit{cm}$ széles és $b=2\unit{cm}$ magas téglalapot. Hálózzuk be a téglalapot $t=0.5\unit{cm}$ oldalhosszúságú négyzet alakú, négy csomópontos síkelemekkel.}

\section{Geometria létrehozása}

Először tároljuk el a szükséges paramétereket:

\comment{Ez a komment nem látszik a dokumentumban.}

\comment{paraméterek}

\code{
a=0.03; 
b=0.02;
t=0.005
}

Elindítjuk a \emph{Preprocessor} folyamatot. Ez szükséges ahhoz, hogy elérhetőek legyenek a következő parancsok.

\comment{preprocessor folyamat}
\code{/PREP7}

Először a sarokpontokat hozzuk létre, majd a téglalap oldalait, végül létrehozzuk a felületet.

\comment{pontok rajzolása}
\ccode{4cm}
{Megrajzoljuk a kulcspontokat. Az első paraméter a pont számát, a további kettő a koordinátáikat jelöli.}
{
K,1,0,0; 
K,2,a,0;
K,3,a,b;
K,4,0,b
}

\comment{vonalak létrehozása}
\ccode{4cm}
{Megrajzoljuk a vonalakat. A két paraméter a kezdő- és végpont sorszáma.}
{
L,1,2;
L,2,3;
L,3,4;
L,4,1;
}

Azért ide gyün még egy kis szöveg.

\comment{felület létrehozása}
\ccode{4cm}{A négy vonal segítségével létrehozzuk a téglalapot.}{AL,1,2,3,4}

El is készítettünk ezt a rendkívül bonyolult geometriát.

\section{Hálózás}

\comment{elemtípus kiválasztása (PLANE 182)}
\ccode{4cm}{Kiválasztjuk az elemtípust, mely a PLANE182 nevű, négy csomópontos kvadratikus síkelem.}
{ET,1,PLANE182}


\code{AESIZE,1,t; MSHAPE,0}

Végül nem marad más dolgunk, mint elvégezni a hálózást és örülni a fejünknek.

\code{AMESH,ALL}

\end{document}